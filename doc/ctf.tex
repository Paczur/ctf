\documentclass[12pt]{article}
\usepackage[T1]{fontenc}
\usepackage{babel}[english]
\usepackage[outputdir=../build/doc, newfloat]{minted}
\usepackage{graphicx}
\usepackage{parskip}
\usepackage{microtype}
\usepackage{float}
\usepackage{txfonts}
\usepackage{fancyvrb}
\usepackage[a4paper, margin=2cm]{geometry}
\graphicspath{ {./img/} }
\usepackage[section]{placeins}
\usepackage[hidelinks]{hyperref}
\usepackage[utf8]{inputenc}
\usepackage{pmboxdraw}
\usepackage{amssymb}
\usepackage{xcolor}
\usepackage[skins,minted]{tcolorbox}
\definecolor{mintedbackground}{rgb}{0.95,0.95,0.95}
\definecolor{mintedframe}{rgb}{0,0,0}
\setminted[c]{
  bgcolor=mintedbackground,
  breaklines=true,
  linenos=true,
  numbersep=5pt,
  frame=leftline,
}
\setminted[bash]{
  bgcolor=mintedbackground,
  breaklines=true,
  linenos=true,
  numbersep=5pt,
  frame=leftline,
  escapeinside=@@,
}
\newtcblisting{shell}[1][]{enhanced, listing engine=minted,
listing only, title=shell, minted language=bash, #1,
coltitle=mintedbackground!30!black,
fonttitle=\ttfamily\footnotesize,
sharp corners, top=0mm, bottom=0mm,
title code={\path[draw=mintedframe,dashed, fill=mintedbackground](title.south west)--(title.south east);},
frame code={\path[draw=mintedframe, fill=mintedbackground](frame.south west) rectangle (frame.north east);}
}
\newtcblisting{ccode}[1][]{enhanced, listing engine=minted,
listing only, #1, minted language=c,
coltitle=mintedbackground!30!black,
fonttitle=\ttfamily\footnotesize,
sharp corners, top=0mm, bottom=0mm,
frame code={\path[draw=mintedframe, fill=mintedbackground](frame.south west) rectangle (frame.north east);}
}
\newtcblisting{ccodefile}[2][]{enhanced, listing engine=minted,
listing only, #1, title=#2, minted language=c,
coltitle=mintedbackground!30!black,
fonttitle=\ttfamily\footnotesize,
sharp corners, top=0mm, bottom=0mm,
title code={\path[draw=mintedframe,dashed, fill=mintedbackground](title.south west)--(title.south east);},
frame code={\path[draw=mintedframe, fill=mintedbackground](frame.south west) rectangle (frame.north east);}
}
\newtcbinputlisting{\cfile}[3][]{enhanced, listing engine=minted,
listing only, listing file=#3,#1, title=#2, minted language=c,
coltitle=mintedbackground!30!black,
fonttitle=\ttfamily\footnotesize,
sharp corners, top=0mm, bottom=0mm,
title code={\path[draw=mintedframe,dashed, fill=mintedbackground](title.south west)--(title.south east);},
frame code={\path[draw=mintedframe, fill=mintedbackground](frame.south west) rectangle (frame.north east);}
}
\title{C Testing Framework Documentation}
\author{v1.0}
\date{ }
\begin{document}
\maketitle
\tableofcontents
\section{Introduction}
C Testing Framework (CTF for short) is framework designed specifically
for ease-of-use. CTF focuses on providing quick feedback loops and test corectness.
It's most notable features are:
\begin{itemize}
  \item Clean and information focused output format.
  \item Rich selection of asserts and expects.
  \item Simplified mocking.
  \item Control of parallel test execution.
\end{itemize}
\textbf{This document assumes familiarity with:}
\begin{itemize}
  \item C language
  \item Unit tests
  \item Any toolchain required for compilation of the C language.
\end{itemize}
\section{Getting Started}
CTF doesn't require specific directory structure. For the purpose of this tuturial, we
will define directory structure as follows:
\begin{verbatim}
src
└──module
   ├──module.c
   └──module.h
test
├──main.c
└──module
   ├──module.c
   └──module.h
\end{verbatim}
Purpose of each directory under \texttt{test} is as follows:
\begin{itemize}
  \item \textbf{\texttt{main.c}: test execution control.}
  \item \texttt{module.c}: tests for \texttt{src/module/module.c}
  \item \texttt{module.h}: external symbols for tests in \texttt{module.c}
\end{itemize}
Foe the purpose of this tutorial, files in \texttt{src} directory are assumed
to have following code:
\cfile{src/module/module.c}{../example/basic/src/module/module.c}
\cfile{src/module/module.h}{../example/basic/src/module/module.h}
\subsection{Your first test}
Defining a test is done with \mintinline{c}{CTF_TEST(name)} macro as follows:
\begin{ccodefile}{test/module/module.c}
CTF_TEST(add_negative) {
  ctf_assert_int_eq(-2, add(-1, -1));
}
\end{ccodefile}
This snippet defines a test named \texttt{add\_negative}.
The test checks if return value of add(-1, -1) equals -2.\\
Using previously defined directory structure;
If function add exists in file \texttt{src/module/module.c},
then above test goes in file \texttt{test/module/module.c}.
\subsection{Grouping tests}
Every test belongs to one or more groups.
Tests not belonging to any group are useless.
Defining a group is done with \mintinline{c}{CTF_GROUP(name)} macro as follows:
\begin{ccodefile}{test/module/module.c}
CTF_GROUP(add_tests) = {
  add_negative,
};
\end{ccodefile}
This snippet defines a group named \texttt{add\_tests}.
Group contains one test defined in previous subsection.\\
Tests frequently belong to only one group. As such this group should be placed in
the same file as test it contains i.e. \texttt{test/module/module.c}.
\subsection{Making group externally visible}
External group declaration is done with \mintinline{c}{CTF_GROUP_EXTERN(name)} macro
as follows:
\begin{ccodefile}{test/module/module.h}
CTF_GROUP_EXTERN(add_tests)
\end{ccodefile}
External symbols should be placed in header corresponding to the source file.\\
In this case \texttt{test/module/module.h}.\\
Note:\\
Tests also can be made externally visible
using \mintinline{c}{CTF_TEST_EXTERNAL(name)} macro.
\subsection{Running grouped tests}
CTF reserves function \mintinline{c}{int main(int argc, char *argv[])} for internal
use.\\
Group execution is done in \mintinline{c}{void ctf_main(int argc, char *argv[])}
function.\\
Running groups of tests is done with either:
\begin{itemize}
  \item \mintinline{c}{ctf_group_run(group)} function
  \item \mintinline{c}{ctf_groups_run(group1, group2, ...)} macro
\end{itemize}
Example:
\begin{ccodefile}{test/main.c}
int ctf_main(int argc, char *argv[]) {
  ctf_group_run(add_tests);
}
\end{ccodefile}
This example runs previously defined group named \texttt{add\_tests}.
CTF exits after \texttt{ctf\_main} function returns.\\
This snippet frequently belongs in \texttt{test/main.c} file.\\
Note:\\
\texttt{test/main.c} should have \mintinline{c}{#include "module/module.h"} clause.
Otherwise \texttt{add\_tests} is undefined symbol.
\subsection{Executing binary}
Compiling above example produces single binary. Executing the binary runs
expected tests and prints their status.\\
Output should look like this:
\begingroup
\catcode`\?=\active
\def?#1{\csname #1\endcsname}
\begin{shell}
  \\[@?{color}{teal}{?{checkmark}}@] add_tests
\end{shell}
\endgroup
By default if all tests in group are sucessful, only group status is shown.
This behaviour can be overriden with "\texttt{-p on}" option.
All available options are listed using \texttt{-h} or \texttt{-{}-help}.
\subsection{Summary}
Key points:
\begin{itemize}
  \item \texttt{CTF\_TEST(name)} defines test.
  \item \texttt{CTF\_GROUP(name)} defines group.
  \item \texttt{CTF\_GROUP(name)} makes group externally visible.
  \item \texttt{ctf\_main} is used instead of \texttt{main}.
  \item \texttt{ctf\_group\_run(group)} runs singular group.
\end{itemize}
All code used in this tutorial:
\cfile{src/module/module.c}{../example/basic/src/module/module.c}
\cfile{src/module/module.h}{../example/basic/src/module/module.h}
\cfile{test/module/module.c}{../example/basic/test/module/module.c}
\cfile{test/module/module.h}{../example/basic/test/module/module.h}
\cfile{test/main.c}{../example/basic/test/main.c}
\section{Guides}
\subsection{Mocking function}
\subsection{Externally visible symbols}
\subsection{Running tests in parallel}
\subsection{Early return on group failure}
\subsection{Redefining SIGSEGV handler}
\subsection{Removing name collisions}
\section{Explanations}
\subsection{Type information in asserts}
\subsection{Running only grouped tests}
\subsection{Print format}
\section{Reference}
\subsection{Tests and groups}
\subsection{Group execution}
\subsection{Mocks}
\subsection{Assertions and expectations}
\subsubsection{Comparisons}
\subsubsection{Special}


\begin{minted}{c}
CTF_TEST(name) { }
CTF_TEST_EXTERN(name);
CTF_GROUP(name) = {}
CTF_GROUP_EXTERN(name);
CTF_GROUP_SETUP(name) {}
CTF_GROUP_TEARDOWN(name) {}
CTF_GROUP_TEST_SETUP(name) {}
CTF_GROUP_TEST_TEARDOWN(name) {}
\end{minted}
\section{Assertions and Expectations}
\subsection{Comparisons}
Available comparison functions:
\begin{minted}{c}
ctf_[a/e]_[type/str]_[cmp](a, b);
ctf_[a/e]_memory_[type]_[cmp](a, b, length);
ctf_[a/e]_array_[type]_[cmp](a, b);
ctf_[a/e]_true(a);
ctf_[a/e]_false(a);
ctf_[a/e]_null(a);
ctf_[a/e]_non_null(a);
\end{minted}
Legend:
\begin{verbatim}
[a/e] = [assert, expect]
[type] = [char, int, uint, ptr]
[type/str] = [char, int, uint, ptr, str]
[cmp] = [eq, neq, gt, gte, lt, lte]
\end{verbatim}
\subsection{Special}
\begin{minted}{c}
ctf_assert_msg(cmp, msg);
ctf_expect_msg(cmp, msg);
ctf_pass(msg);
ctf_fail(msg);
ctf_assert_fold(count, msg);
ctf_assert_barrier();
\end{minted}
\section{Control Flow}
\begin{minted}{c}
ctf_sigsegv_handler(unused);
ctf_parallel_start();
ctf_parallel_stop();
ctf_parallel_sync();
ctf_barrier();
ctf_group_run(group);
ctf_groups_run(group1, group2, ...)
int ctf_exit_code;
int ctf_entry_point(argc, argv) {}
\end{minted}
\section{Mocks}
\begin{minted}{c}
CTF_MOCK(ret, name, typed_args, args)
CTF_MOCK_VOID_RET(name, typed_args, args)
CTF_MOCK_EXTERN(ret_type, name, typed_args)
CTF_MOCK_COID_RET_EXTERN(ret_type, name, typed_args)
CTF_MOCK_GROUP(name) = { CTF_MOCK_BIND(fn, mock) }
CTF_MOCK_GROUP_EXTERN(name)

ctf_mock_[a/e]_[type/str]_[cmp](a, b);
ctf_mock_[a/e]_memory_[type]_[cmp](a, b, length);
ctf_mock_[a/e]_array_[type]_[cmp](a, b);
ctf_mock_[a/e]_true(a);
ctf_mock_[a/e]_false(a);
ctf_mock_[a/e]_null(a);
ctf_mock_[a/e]_non_null(a);

ctf_mock(mock, fn);
ctf_unmock(mock);
ctf_mock_group(group);
ctf_mock_ungroup(group);
ctf_mock_call_count(mock);
ctf_mock_will_return(mock, val);
ctf_mock_will_return_once(mock, val);
ctf_mock_real(mock);
\end{minted}
Legend:
\begin{verbatim}
[a/e] = [assert, expect]
[type] = [char, int, uint, ptr]
[type/str] = [char, int, uint, ptr, str]
[cmp] = [eq, neq, gt, gte, lt, lte]
\end{verbatim}
\end{document}
